%% Generated by Sphinx.
\def\sphinxdocclass{report}
\documentclass[letterpaper,10pt,polish]{sphinxmanual}
\ifdefined\pdfpxdimen
   \let\sphinxpxdimen\pdfpxdimen\else\newdimen\sphinxpxdimen
\fi \sphinxpxdimen=.75bp\relax
\ifdefined\pdfimageresolution
    \pdfimageresolution= \numexpr \dimexpr1in\relax/\sphinxpxdimen\relax
\fi
%% let collapsible pdf bookmarks panel have high depth per default
\PassOptionsToPackage{bookmarksdepth=5}{hyperref}

\PassOptionsToPackage{booktabs}{sphinx}
\PassOptionsToPackage{colorrows}{sphinx}

\PassOptionsToPackage{warn}{textcomp}
\usepackage[utf8]{inputenc}
\ifdefined\DeclareUnicodeCharacter
% support both utf8 and utf8x syntaxes
  \ifdefined\DeclareUnicodeCharacterAsOptional
    \def\sphinxDUC#1{\DeclareUnicodeCharacter{"#1}}
  \else
    \let\sphinxDUC\DeclareUnicodeCharacter
  \fi
  \sphinxDUC{00A0}{\nobreakspace}
  \sphinxDUC{2500}{\sphinxunichar{2500}}
  \sphinxDUC{2502}{\sphinxunichar{2502}}
  \sphinxDUC{2514}{\sphinxunichar{2514}}
  \sphinxDUC{251C}{\sphinxunichar{251C}}
  \sphinxDUC{2572}{\textbackslash}
\fi
\usepackage{cmap}
\usepackage[T1]{fontenc}
\usepackage{amsmath,amssymb,amstext}
\usepackage{babel}



\usepackage{tgtermes}
\usepackage{tgheros}
\renewcommand{\ttdefault}{txtt}



\usepackage[Sonny]{fncychap}
\ChNameVar{\Large\normalfont\sffamily}
\ChTitleVar{\Large\normalfont\sffamily}
\usepackage{sphinx}

\fvset{fontsize=auto}
\usepackage{geometry}


% Include hyperref last.
\usepackage{hyperref}
% Fix anchor placement for figures with captions.
\usepackage{hypcap}% it must be loaded after hyperref.
% Set up styles of URL: it should be placed after hyperref.
\urlstyle{same}

\addto\captionspolish{\renewcommand{\contentsname}{Spis treści:}}

\usepackage{sphinxmessages}
\setcounter{tocdepth}{1}



\title{Kopie zapasowe i odzyskiwanie danych}
\date{12 cze 2025}
\release{1.0}
\author{Miłosz Śmieja}
\newcommand{\sphinxlogo}{\vbox{}}
\renewcommand{\releasename}{Wydanie}
\makeindex
\begin{document}

\ifdefined\shorthandoff
  \ifnum\catcode`\=\string=\active\shorthandoff{=}\fi
  \ifnum\catcode`\"=\active\shorthandoff{"}\fi
\fi

\pagestyle{empty}
\sphinxmaketitle
\pagestyle{plain}
\sphinxtableofcontents
\pagestyle{normal}
\phantomsection\label{\detokenize{index::doc}}


\sphinxstepscope


\chapter{Kopie zapasowe i odzyskiwanie danych w PostgreSQL}
\label{\detokenize{kopie_zapasowe_i_odzyskiwanie_danych:kopie-zapasowe-i-odzyskiwanie-danych-w-postgresql}}\label{\detokenize{kopie_zapasowe_i_odzyskiwanie_danych::doc}}\begin{quote}\begin{description}
\sphinxlineitem{Autor}
\sphinxAtStartPar
Miłosz Śmieja

\sphinxlineitem{Wersja}
\sphinxAtStartPar
1.0

\sphinxlineitem{Data}
\sphinxAtStartPar
12 czerwca 2025

\end{description}\end{quote}

\begin{sphinxShadowBox}
\sphinxstyletopictitle{Spis treści}
\begin{itemize}
\item {} 
\sphinxAtStartPar
\phantomsection\label{\detokenize{kopie_zapasowe_i_odzyskiwanie_danych:id1}}{\hyperref[\detokenize{kopie_zapasowe_i_odzyskiwanie_danych:wprowadzenie}]{\sphinxcrossref{Wprowadzenie}}}

\item {} 
\sphinxAtStartPar
\phantomsection\label{\detokenize{kopie_zapasowe_i_odzyskiwanie_danych:id2}}{\hyperref[\detokenize{kopie_zapasowe_i_odzyskiwanie_danych:mechanizmy-wbudowane-do-tworzenia-kopii-zapasowych-calego-systemu-postgresql}]{\sphinxcrossref{Mechanizmy wbudowane do tworzenia kopii zapasowych całego systemu PostgreSQL}}}
\begin{itemize}
\item {} 
\sphinxAtStartPar
\phantomsection\label{\detokenize{kopie_zapasowe_i_odzyskiwanie_danych:id3}}{\hyperref[\detokenize{kopie_zapasowe_i_odzyskiwanie_danych:pg-basebackup}]{\sphinxcrossref{pg\_basebackup}}}

\item {} 
\sphinxAtStartPar
\phantomsection\label{\detokenize{kopie_zapasowe_i_odzyskiwanie_danych:id4}}{\hyperref[\detokenize{kopie_zapasowe_i_odzyskiwanie_danych:continuous-archiving-point-in-time-recovery}]{\sphinxcrossref{Continuous Archiving (Point\sphinxhyphen{}in\sphinxhyphen{}Time Recovery)}}}

\item {} 
\sphinxAtStartPar
\phantomsection\label{\detokenize{kopie_zapasowe_i_odzyskiwanie_danych:id5}}{\hyperref[\detokenize{kopie_zapasowe_i_odzyskiwanie_danych:streaming-replication}]{\sphinxcrossref{Streaming Replication}}}

\item {} 
\sphinxAtStartPar
\phantomsection\label{\detokenize{kopie_zapasowe_i_odzyskiwanie_danych:id6}}{\hyperref[\detokenize{kopie_zapasowe_i_odzyskiwanie_danych:file-system-level-backup}]{\sphinxcrossref{File System Level Backup}}}

\end{itemize}

\item {} 
\sphinxAtStartPar
\phantomsection\label{\detokenize{kopie_zapasowe_i_odzyskiwanie_danych:id7}}{\hyperref[\detokenize{kopie_zapasowe_i_odzyskiwanie_danych:mechanizmy-wbudowane-do-tworzenia-kopii-zapasowych-poszczegolnych-baz-danych}]{\sphinxcrossref{Mechanizmy wbudowane do tworzenia kopii zapasowych poszczególnych baz danych}}}
\begin{itemize}
\item {} 
\sphinxAtStartPar
\phantomsection\label{\detokenize{kopie_zapasowe_i_odzyskiwanie_danych:id8}}{\hyperref[\detokenize{kopie_zapasowe_i_odzyskiwanie_danych:pg-dump}]{\sphinxcrossref{pg\_dump}}}

\item {} 
\sphinxAtStartPar
\phantomsection\label{\detokenize{kopie_zapasowe_i_odzyskiwanie_danych:id9}}{\hyperref[\detokenize{kopie_zapasowe_i_odzyskiwanie_danych:pg-dumpall}]{\sphinxcrossref{pg\_dumpall}}}

\item {} 
\sphinxAtStartPar
\phantomsection\label{\detokenize{kopie_zapasowe_i_odzyskiwanie_danych:id10}}{\hyperref[\detokenize{kopie_zapasowe_i_odzyskiwanie_danych:copy-command}]{\sphinxcrossref{COPY command}}}

\item {} 
\sphinxAtStartPar
\phantomsection\label{\detokenize{kopie_zapasowe_i_odzyskiwanie_danych:id11}}{\hyperref[\detokenize{kopie_zapasowe_i_odzyskiwanie_danych:pg-dump-z-opcjami-selektywnymi}]{\sphinxcrossref{pg\_dump z opcjami selektywnymi}}}

\end{itemize}

\item {} 
\sphinxAtStartPar
\phantomsection\label{\detokenize{kopie_zapasowe_i_odzyskiwanie_danych:id12}}{\hyperref[\detokenize{kopie_zapasowe_i_odzyskiwanie_danych:odzyskiwanie-usunietych-lub-uszkodzonych-danych}]{\sphinxcrossref{Odzyskiwanie usuniętych lub uszkodzonych danych}}}
\begin{itemize}
\item {} 
\sphinxAtStartPar
\phantomsection\label{\detokenize{kopie_zapasowe_i_odzyskiwanie_danych:id13}}{\hyperref[\detokenize{kopie_zapasowe_i_odzyskiwanie_danych:odzyskiwanie-z-kopii-logicznych}]{\sphinxcrossref{Odzyskiwanie z kopii logicznych}}}

\item {} 
\sphinxAtStartPar
\phantomsection\label{\detokenize{kopie_zapasowe_i_odzyskiwanie_danych:id14}}{\hyperref[\detokenize{kopie_zapasowe_i_odzyskiwanie_danych:point-in-time-recovery-pitr}]{\sphinxcrossref{Point\sphinxhyphen{}in\sphinxhyphen{}Time Recovery (PITR)}}}

\item {} 
\sphinxAtStartPar
\phantomsection\label{\detokenize{kopie_zapasowe_i_odzyskiwanie_danych:id15}}{\hyperref[\detokenize{kopie_zapasowe_i_odzyskiwanie_danych:odzyskiwanie-tabel-z-tablespaces}]{\sphinxcrossref{Odzyskiwanie tabel z tablespaces}}}

\item {} 
\sphinxAtStartPar
\phantomsection\label{\detokenize{kopie_zapasowe_i_odzyskiwanie_danych:id16}}{\hyperref[\detokenize{kopie_zapasowe_i_odzyskiwanie_danych:transaction-log-replay}]{\sphinxcrossref{Transaction log replay}}}

\item {} 
\sphinxAtStartPar
\phantomsection\label{\detokenize{kopie_zapasowe_i_odzyskiwanie_danych:id17}}{\hyperref[\detokenize{kopie_zapasowe_i_odzyskiwanie_danych:odzyskiwanie-na-poziomie-klastra}]{\sphinxcrossref{Odzyskiwanie na poziomie klastra}}}

\end{itemize}

\item {} 
\sphinxAtStartPar
\phantomsection\label{\detokenize{kopie_zapasowe_i_odzyskiwanie_danych:id18}}{\hyperref[\detokenize{kopie_zapasowe_i_odzyskiwanie_danych:dedykowane-oprogramowanie-i-skrypty-zewnetrzne-do-automatyzacji}]{\sphinxcrossref{Dedykowane oprogramowanie i skrypty zewnętrzne do automatyzacji}}}
\begin{itemize}
\item {} 
\sphinxAtStartPar
\phantomsection\label{\detokenize{kopie_zapasowe_i_odzyskiwanie_danych:id19}}{\hyperref[\detokenize{kopie_zapasowe_i_odzyskiwanie_danych:pgbackrest}]{\sphinxcrossref{pgBackRest}}}

\item {} 
\sphinxAtStartPar
\phantomsection\label{\detokenize{kopie_zapasowe_i_odzyskiwanie_danych:id20}}{\hyperref[\detokenize{kopie_zapasowe_i_odzyskiwanie_danych:barman-backup-and-recovery-manager}]{\sphinxcrossref{Barman (Backup and Recovery Manager)}}}

\item {} 
\sphinxAtStartPar
\phantomsection\label{\detokenize{kopie_zapasowe_i_odzyskiwanie_danych:id21}}{\hyperref[\detokenize{kopie_zapasowe_i_odzyskiwanie_danych:wal-e-i-wal-g}]{\sphinxcrossref{WAL\sphinxhyphen{}E i WAL\sphinxhyphen{}G}}}

\item {} 
\sphinxAtStartPar
\phantomsection\label{\detokenize{kopie_zapasowe_i_odzyskiwanie_danych:id22}}{\hyperref[\detokenize{kopie_zapasowe_i_odzyskiwanie_danych:skrypty-shell-i-cron-jobs}]{\sphinxcrossref{Skrypty shell i cron jobs}}}

\item {} 
\sphinxAtStartPar
\phantomsection\label{\detokenize{kopie_zapasowe_i_odzyskiwanie_danych:id23}}{\hyperref[\detokenize{kopie_zapasowe_i_odzyskiwanie_danych:narzedzia-automatyzacji-infrastruktury}]{\sphinxcrossref{Narzędzia automatyzacji infrastruktury}}}

\item {} 
\sphinxAtStartPar
\phantomsection\label{\detokenize{kopie_zapasowe_i_odzyskiwanie_danych:id24}}{\hyperref[\detokenize{kopie_zapasowe_i_odzyskiwanie_danych:monitoring-i-alertowanie}]{\sphinxcrossref{Monitoring i alertowanie}}}

\end{itemize}

\item {} 
\sphinxAtStartPar
\phantomsection\label{\detokenize{kopie_zapasowe_i_odzyskiwanie_danych:id25}}{\hyperref[\detokenize{kopie_zapasowe_i_odzyskiwanie_danych:podsumowanie}]{\sphinxcrossref{Podsumowanie}}}
\begin{itemize}
\item {} 
\sphinxAtStartPar
\phantomsection\label{\detokenize{kopie_zapasowe_i_odzyskiwanie_danych:id26}}{\hyperref[\detokenize{kopie_zapasowe_i_odzyskiwanie_danych:kluczowe-wnioski}]{\sphinxcrossref{Kluczowe wnioski}}}

\item {} 
\sphinxAtStartPar
\phantomsection\label{\detokenize{kopie_zapasowe_i_odzyskiwanie_danych:id27}}{\hyperref[\detokenize{kopie_zapasowe_i_odzyskiwanie_danych:najwazniejsze-zalecenia}]{\sphinxcrossref{Najważniejsze zalecenia}}}

\end{itemize}

\end{itemize}
\end{sphinxShadowBox}


\section{Wprowadzenie}
\label{\detokenize{kopie_zapasowe_i_odzyskiwanie_danych:wprowadzenie}}
\sphinxAtStartPar
System zarządzania bazą danych PostgreSQL oferuje kompleksowy zestaw narzędzi i mechanizmów służących do tworzenia kopii zapasowych oraz odzyskiwania danych. Skuteczne zarządzanie kopiami zapasowymi stanowi fundament bezpieczeństwa danych i ciągłości działania systemów bazodanowych.

\sphinxAtStartPar
PostgreSQL dostarcza zarówno mechanizmy wbudowane, jak i możliwość integracji z zewnętrznymi narzędziami automatyzacji.


\section{Mechanizmy wbudowane do tworzenia kopii zapasowych całego systemu PostgreSQL}
\label{\detokenize{kopie_zapasowe_i_odzyskiwanie_danych:mechanizmy-wbudowane-do-tworzenia-kopii-zapasowych-calego-systemu-postgresql}}
\sphinxAtStartPar
PostgreSQL oferuje kilka mechanizmów tworzenia kopii zapasowych na poziomie całego systemu, które zapewniają kompleksową ochronę wszystkich baz danych w klastrze.


\subsection{pg\_basebackup}
\label{\detokenize{kopie_zapasowe_i_odzyskiwanie_danych:pg-basebackup}}
\sphinxAtStartPar
\sphinxstylestrong{pg\_basebackup} stanowi podstawowe narzędzie do tworzenia fizycznych kopii zapasowych całego klastra PostgreSQL.

\sphinxAtStartPar
Kluczowe cechy:
\begin{itemize}
\item {} 
\sphinxAtStartPar
Działa w trybie online \sphinxhyphen{} możliwość wykonywania kopii zapasowych bez zatrzymywania działania serwera

\item {} 
\sphinxAtStartPar
Tworzy dokładną kopię wszystkich plików danych

\item {} 
\sphinxAtStartPar
Zawiera pliki konfiguracyjne, dzienniki transakcji oraz wszystkie bazy danych w klastrze

\end{itemize}


\subsection{Continuous Archiving (Point\sphinxhyphen{}in\sphinxhyphen{}Time Recovery)}
\label{\detokenize{kopie_zapasowe_i_odzyskiwanie_danych:continuous-archiving-point-in-time-recovery}}
\sphinxAtStartPar
\sphinxstylestrong{Continuous Archiving} reprezentuje zaawansowany mechanizm tworzenia ciągłych kopii zapasowych poprzez archiwizację dzienników WAL (Write\sphinxhyphen{}Ahead Logging).

\sphinxAtStartPar
Zalety:
\begin{itemize}
\item {} 
\sphinxAtStartPar
Umożliwia odtworzenie stanu bazy danych w dowolnym momencie czasowym

\item {} 
\sphinxAtStartPar
Szczególnie wartościowe w środowiskach produkcyjnych wymagających minimalnej utraty danych

\item {} 
\sphinxAtStartPar
Zapewnia wysoką granularność odzyskiwania danych

\end{itemize}


\subsection{Streaming Replication}
\label{\detokenize{kopie_zapasowe_i_odzyskiwanie_danych:streaming-replication}}
\sphinxAtStartPar
\sphinxstylestrong{Streaming Replication} może służyć jako mechanizm kopii zapasowych poprzez utrzymywanie synchronicznych lub asynchronicznych replik głównej bazy danych.

\sphinxAtStartPar
Funkcjonalności:
\begin{itemize}
\item {} 
\sphinxAtStartPar
Repliki funkcjonują jako kopie zapasowe w czasie rzeczywistym

\item {} 
\sphinxAtStartPar
Oferuje możliwość szybkiego przełączenia w przypadku awarii systemu głównego

\item {} 
\sphinxAtStartPar
Wspiera zarówno tryb synchroniczny, jak i asynchroniczny

\end{itemize}


\subsection{File System Level Backup}
\label{\detokenize{kopie_zapasowe_i_odzyskiwanie_danych:file-system-level-backup}}
\sphinxAtStartPar
\sphinxstylestrong{File System Level Backup} polega na tworzeniu kopii zapasowych na poziomie systemu plików.

\sphinxAtStartPar
Wymagania:
\begin{itemize}
\item {} 
\sphinxAtStartPar
Zatrzymanie serwera PostgreSQL lub zapewnienie spójności

\item {} 
\sphinxAtStartPar
Wykorzystanie mechanizmów snapshot systemu plików:
\begin{itemize}
\item {} 
\sphinxAtStartPar
LVM snapshots

\item {} 
\sphinxAtStartPar
ZFS snapshots

\end{itemize}

\end{itemize}


\section{Mechanizmy wbudowane do tworzenia kopii zapasowych poszczególnych baz danych}
\label{\detokenize{kopie_zapasowe_i_odzyskiwanie_danych:mechanizmy-wbudowane-do-tworzenia-kopii-zapasowych-poszczegolnych-baz-danych}}
\sphinxAtStartPar
PostgreSQL dostarcza precyzyjne narzędzia umożliwiające tworzenie kopii zapasowych pojedynczych baz danych lub ich wybranych elementów.


\subsection{pg\_dump}
\label{\detokenize{kopie_zapasowe_i_odzyskiwanie_danych:pg-dump}}
\sphinxAtStartPar
\sphinxstylestrong{pg\_dump} stanowi najczęściej wykorzystywane narzędzie do tworzenia logicznych kopii zapasowych pojedynczych baz danych.

\sphinxAtStartPar
Charakterystyka:
\begin{itemize}
\item {} 
\sphinxAtStartPar
Tworzy skrypt SQL zawierający wszystkie polecenia niezbędne do odtworzenia struktury bazy danych oraz jej danych

\item {} 
\sphinxAtStartPar
Oferuje liczne opcje konfiguracji:
\begin{itemize}
\item {} 
\sphinxAtStartPar
Możliwość wyboru formatu wyjściowego

\item {} 
\sphinxAtStartPar
Filtrowanie obiektów

\item {} 
\sphinxAtStartPar
Kontrola nad poziomem szczegółowości kopii zapasowej

\end{itemize}

\end{itemize}


\subsection{pg\_dumpall}
\label{\detokenize{kopie_zapasowe_i_odzyskiwanie_danych:pg-dumpall}}
\sphinxAtStartPar
\sphinxstylestrong{pg\_dumpall} rozszerza funkcjonalność \sphinxcode{\sphinxupquote{pg\_dump}} o możliwość tworzenia kopii zapasowych wszystkich baz danych w klastrze.

\sphinxAtStartPar
Dodatkowe funkcje:
\begin{itemize}
\item {} 
\sphinxAtStartPar
Backup obiektów globalnych:
\begin{itemize}
\item {} 
\sphinxAtStartPar
Role użytkowników

\item {} 
\sphinxAtStartPar
Tablespaces

\item {} 
\sphinxAtStartPar
Ustawienia konfiguracyjne na poziomie klastra

\end{itemize}

\end{itemize}


\subsection{COPY command}
\label{\detokenize{kopie_zapasowe_i_odzyskiwanie_danych:copy-command}}
\sphinxAtStartPar
\sphinxstylestrong{COPY command} umożliwia eksport danych z poszczególnych tabel do plików w różnych formatach.

\sphinxAtStartPar
Obsługiwane formaty:
\begin{itemize}
\item {} 
\sphinxAtStartPar
CSV

\item {} 
\sphinxAtStartPar
Text

\item {} 
\sphinxAtStartPar
Binary

\end{itemize}

\sphinxAtStartPar
Zastosowania:
\begin{itemize}
\item {} 
\sphinxAtStartPar
Tworzenie selektywnych kopii zapasowych dużych tabel

\item {} 
\sphinxAtStartPar
Migracje danych

\end{itemize}


\subsection{pg\_dump z opcjami selektywnymi}
\label{\detokenize{kopie_zapasowe_i_odzyskiwanie_danych:pg-dump-z-opcjami-selektywnymi}}
\sphinxAtStartPar
\sphinxstylestrong{pg\_dump z opcjami selektywnymi} pozwala na tworzenie kopii zapasowych wybranych obiektów bazy danych.

\sphinxAtStartPar
Możliwości filtrowania:
\begin{itemize}
\item {} 
\sphinxAtStartPar
Konkretne tabele

\item {} 
\sphinxAtStartPar
Schematy

\item {} 
\sphinxAtStartPar
Sekwencje

\end{itemize}

\sphinxAtStartPar
Funkcjonalność ta jest nieoceniona w scenariuszach wymagających granularnej kontroli nad procesem tworzenia kopii zapasowych.


\section{Odzyskiwanie usuniętych lub uszkodzonych danych}
\label{\detokenize{kopie_zapasowe_i_odzyskiwanie_danych:odzyskiwanie-usunietych-lub-uszkodzonych-danych}}
\sphinxAtStartPar
PostgreSQL oferuje różnorodne mechanizmy odzyskiwania danych w zależności od rodzaju i zakresu uszkodzeń.


\subsection{Odzyskiwanie z kopii logicznych}
\label{\detokenize{kopie_zapasowe_i_odzyskiwanie_danych:odzyskiwanie-z-kopii-logicznych}}
\sphinxAtStartPar
\sphinxstylestrong{Odzyskiwanie z kopii logicznych} wykonanych przy użyciu \sphinxcode{\sphinxupquote{pg\_dump}} realizowane jest poprzez \sphinxcode{\sphinxupquote{psql}} lub \sphinxcode{\sphinxupquote{pg\_restore}}.

\sphinxAtStartPar
Proces odzyskiwania:
\begin{itemize}
\item {} 
\sphinxAtStartPar
Wykonanie skryptów SQL

\item {} 
\sphinxAtStartPar
Przywrócenie plików dump w odpowiednim formacie

\end{itemize}

\sphinxAtStartPar
Zaawansowane opcje pg\_restore:
\begin{itemize}
\item {} 
\sphinxAtStartPar
Selektywne przywracanie obiektów

\item {} 
\sphinxAtStartPar
Równoległe przetwarzanie

\item {} 
\sphinxAtStartPar
Kontrola nad kolejnością przywracania

\end{itemize}


\subsection{Point\sphinxhyphen{}in\sphinxhyphen{}Time Recovery (PITR)}
\label{\detokenize{kopie_zapasowe_i_odzyskiwanie_danych:point-in-time-recovery-pitr}}
\sphinxAtStartPar
\sphinxstylestrong{Point\sphinxhyphen{}in\sphinxhyphen{}Time Recovery (PITR)} umożliwia przywrócenie bazy danych do konkretnego momentu w czasie.

\sphinxAtStartPar
Wykorzystywane komponenty:
\begin{itemize}
\item {} 
\sphinxAtStartPar
Kombinacja kopii bazowej

\item {} 
\sphinxAtStartPar
Archiwalne dzienniki WAL

\end{itemize}

\sphinxAtStartPar
Zastosowania:
\begin{itemize}
\item {} 
\sphinxAtStartPar
Cofnięcie zmian do momentu poprzedzającego wystąpienie błędu

\item {} 
\sphinxAtStartPar
Odzyskiwanie po uszkodzeniu danych

\end{itemize}

\begin{sphinxadmonition}{note}{Informacja:}
\sphinxAtStartPar
PITR jest szczególnie wartościowy w przypadkach, gdy konieczne jest cofnięcie zmian do momentu poprzedzającego wystąpienie błędu lub uszkodzenia.
\end{sphinxadmonition}


\subsection{Odzyskiwanie tabel z tablespaces}
\label{\detokenize{kopie_zapasowe_i_odzyskiwanie_danych:odzyskiwanie-tabel-z-tablespaces}}
\sphinxAtStartPar
\sphinxstylestrong{Odzyskiwanie tabel z tablespaces} może wymagać specjalnych procedur w przypadku uszkodzenia przestrzeni tabel.

\sphinxAtStartPar
Możliwości PostgreSQL:
\begin{itemize}
\item {} 
\sphinxAtStartPar
Odtworzenie tablespaces

\item {} 
\sphinxAtStartPar
Przeniesienie tabel między różnymi lokalizacjami

\item {} 
\sphinxAtStartPar
Odzyskiwanie danych nawet w przypadku częściowego uszkodzenia systemu plików

\end{itemize}


\subsection{Transaction log replay}
\label{\detokenize{kopie_zapasowe_i_odzyskiwanie_danych:transaction-log-replay}}
\sphinxAtStartPar
\sphinxstylestrong{Transaction log replay} wykorzystuje dzienniki WAL do odtworzenia zmian wprowadzonych po utworzeniu kopii zapasowej.

\sphinxAtStartPar
Charakterystyka:
\begin{itemize}
\item {} 
\sphinxAtStartPar
Automatycznie wykorzystywany podczas standardowych procedur odzyskiwania

\item {} 
\sphinxAtStartPar
Możliwość ręcznej kontroli w szczególnych sytuacjach

\end{itemize}


\subsection{Odzyskiwanie na poziomie klastra}
\label{\detokenize{kopie_zapasowe_i_odzyskiwanie_danych:odzyskiwanie-na-poziomie-klastra}}
\sphinxAtStartPar
\sphinxstylestrong{Odzyskiwanie na poziomie klastra} przy wykorzystaniu \sphinxcode{\sphinxupquote{pg\_basebackup}} wymaga przywrócenia wszystkich plików klastra oraz odpowiedniej konfiguracji parametrów recovery.

\sphinxAtStartPar
Zakres procesu:
\begin{itemize}
\item {} 
\sphinxAtStartPar
Odtworzenie całego środowiska PostgreSQL

\item {} 
\sphinxAtStartPar
Konfiguracja ról i uprawnień

\item {} 
\sphinxAtStartPar
Przywrócenie ustawień systemowych

\end{itemize}


\section{Dedykowane oprogramowanie i skrypty zewnętrzne do automatyzacji}
\label{\detokenize{kopie_zapasowe_i_odzyskiwanie_danych:dedykowane-oprogramowanie-i-skrypty-zewnetrzne-do-automatyzacji}}
\sphinxAtStartPar
Automatyzacja procesów tworzenia kopii zapasowych stanowi kluczowy element profesjonalnego zarządzania bazami danych PostgreSQL.


\subsection{pgBackRest}
\label{\detokenize{kopie_zapasowe_i_odzyskiwanie_danych:pgbackrest}}
\sphinxAtStartPar
\sphinxstylestrong{pgBackRest} reprezentuje kompleksowe rozwiązanie do zarządzania kopiami zapasowymi PostgreSQL.

\sphinxAtStartPar
Zaawansowane funkcje:
\begin{itemize}
\item {} 
\sphinxAtStartPar
Incremental i differential backups

\item {} 
\sphinxAtStartPar
Kompresja danych

\item {} 
\sphinxAtStartPar
Szyfrowanie

\item {} 
\sphinxAtStartPar
Weryfikacja integralności kopii

\item {} 
\sphinxAtStartPar
Możliwość przechowywania kopii w chmurze

\item {} 
\sphinxAtStartPar
Automatyzacja procesów zarządzania kopiami zapasowymi

\item {} 
\sphinxAtStartPar
Uproszczone procedury odzyskiwania

\end{itemize}

\begin{sphinxadmonition}{important}{Ważne:}
\sphinxAtStartPar
pgBackRest automatyzuje wiele procesów związanych z zarządzaniem kopiami zapasowymi i znacznie upraszcza procedury odzyskiwania.
\end{sphinxadmonition}


\subsection{Barman (Backup and Recovery Manager)}
\label{\detokenize{kopie_zapasowe_i_odzyskiwanie_danych:barman-backup-and-recovery-manager}}
\sphinxAtStartPar
\sphinxstylestrong{Barman} stanowi dedykowane narzędzie stworzone przez 2ndQuadrant do zarządzania kopiami zapasowymi PostgreSQL w środowiskach enterprise.

\sphinxAtStartPar
Kluczowe funkcjonalności:
\begin{itemize}
\item {} 
\sphinxAtStartPar
Centralne zarządzanie kopiami zapasowymi wielu serwerów PostgreSQL

\item {} 
\sphinxAtStartPar
Monitoring procesów backup

\item {} 
\sphinxAtStartPar
Automatyczne testowanie procedur recovery

\item {} 
\sphinxAtStartPar
Integracja z narzędziami monitorowania

\end{itemize}


\subsection{WAL\sphinxhyphen{}E i WAL\sphinxhyphen{}G}
\label{\detokenize{kopie_zapasowe_i_odzyskiwanie_danych:wal-e-i-wal-g}}
\sphinxAtStartPar
\sphinxstylestrong{WAL\sphinxhyphen{}E i WAL\sphinxhyphen{}G} specjalizują się w archiwizacji dzienników WAL w środowiskach chmurowych.

\sphinxAtStartPar
Oferowane funkcje:
\begin{itemize}
\item {} 
\sphinxAtStartPar
Efektywna kompresja

\item {} 
\sphinxAtStartPar
Szyfrowanie danych

\item {} 
\sphinxAtStartPar
Przechowywanie kopii zapasowych w serwisach chmurowych:
\begin{itemize}
\item {} 
\sphinxAtStartPar
Amazon S3

\item {} 
\sphinxAtStartPar
Google Cloud Storage

\item {} 
\sphinxAtStartPar
Azure Blob Storage

\end{itemize}

\end{itemize}


\subsection{Skrypty shell i cron jobs}
\label{\detokenize{kopie_zapasowe_i_odzyskiwanie_danych:skrypty-shell-i-cron-jobs}}
\sphinxAtStartPar
\sphinxstylestrong{Skrypty shell i cron jobs} stanowią tradycyjne podejście do automatyzacji kopii zapasowych.

\sphinxAtStartPar
Możliwości automatyzacji:
\begin{itemize}
\item {} 
\sphinxAtStartPar
Wykonywanie \sphinxcode{\sphinxupquote{pg\_dump}} i \sphinxcode{\sphinxupquote{pg\_basebackup}}

\item {} 
\sphinxAtStartPar
Zarządzanie cyklem życia kopii zapasowych

\item {} 
\sphinxAtStartPar
Rotacja i czyszczenie starych kopii

\end{itemize}

\begin{sphinxadmonition}{tip}{Wskazówka:}
\sphinxAtStartPar
Właściwie napisane skrypty mogą automatyzować wykonywanie pg\_dump, pg\_basebackup oraz zarządzanie cyklem życia kopii zapasowych, w tym rotację i czyszczenie starych kopii.
\end{sphinxadmonition}


\subsection{Narzędzia automatyzacji infrastruktury}
\label{\detokenize{kopie_zapasowe_i_odzyskiwanie_danych:narzedzia-automatyzacji-infrastruktury}}
\sphinxAtStartPar
\sphinxstylestrong{Ansible, Puppet, Chef} jako narzędzia automatyzacji infrastruktury mogą być wykorzystywane do zarządzania konfiguracją procesów backup na większą skalę.

\sphinxAtStartPar
Korzyści:
\begin{itemize}
\item {} 
\sphinxAtStartPar
Standaryzacja procedur backup w środowiskach wieloserwerowych

\item {} 
\sphinxAtStartPar
Zapewnienie konsystentności konfiguracji

\item {} 
\sphinxAtStartPar
Skalowalne zarządzanie infrastrukturą

\end{itemize}


\subsection{Monitoring i alertowanie}
\label{\detokenize{kopie_zapasowe_i_odzyskiwanie_danych:monitoring-i-alertowanie}}
\sphinxAtStartPar
\sphinxstylestrong{Prometheus i Grafana} w połączeniu z \sphinxcode{\sphinxupquote{postgres\_exporter}} umożliwiają monitoring procesów backup oraz alertowanie w przypadku niepowodzeń.

\sphinxAtStartPar
Zakres monitorowania:
\begin{itemize}
\item {} 
\sphinxAtStartPar
Śledzenie czasu wykonywania kopii

\item {} 
\sphinxAtStartPar
Monitorowanie rozmiaru kopii zapasowych

\item {} 
\sphinxAtStartPar
Wskaźnik sukcesu procesów backup

\item {} 
\sphinxAtStartPar
Alertowanie w czasie rzeczywistym

\end{itemize}


\section{Podsumowanie}
\label{\detokenize{kopie_zapasowe_i_odzyskiwanie_danych:podsumowanie}}
\sphinxAtStartPar
Skuteczne zarządzanie kopiami zapasowymi w PostgreSQL wymaga kombinacji mechanizmów wbudowanych oraz zewnętrznych narzędzi automatyzacji. Wybór odpowiedniej strategii backup zależy od specyficznych wymagań organizacji, w tym:
\begin{itemize}
\item {} 
\sphinxAtStartPar
\sphinxstylestrong{RTO (Recovery Time Objective)} \sphinxhyphen{} maksymalny akceptowalny czas odzyskiwania

\item {} 
\sphinxAtStartPar
\sphinxstylestrong{RPO (Recovery Point Objective)} \sphinxhyphen{} maksymalna akceptowalna utrata danych

\item {} 
\sphinxAtStartPar
Dostępne zasoby

\item {} 
\sphinxAtStartPar
Złożoność środowiska

\end{itemize}


\subsection{Kluczowe wnioski}
\label{\detokenize{kopie_zapasowe_i_odzyskiwanie_danych:kluczowe-wnioski}}
\sphinxAtStartPar
\sphinxstylestrong{Mechanizmy wbudowane} PostgreSQL, takie jak \sphinxcode{\sphinxupquote{pg\_dump}}, \sphinxcode{\sphinxupquote{pg\_basebackup}} czy PITR, oferują solidne podstawy dla większości scenariuszy backup i recovery.

\sphinxAtStartPar
\sphinxstylestrong{W środowiskach produkcyjnych} o wysokich wymaganiach dotyczących dostępności i niezawodności, integracja z dedykowanymi narzędziami takimi jak pgBackRest czy Barman staje się niezbędna.


\subsection{Najważniejsze zalecenia}
\label{\detokenize{kopie_zapasowe_i_odzyskiwanie_danych:najwazniejsze-zalecenia}}
\begin{sphinxadmonition}{warning}{Ostrzeżenie:}
\sphinxAtStartPar
Kluczowym elementem każdej strategii backup jest regularne testowanie procedur odzyskiwania danych. Kopie zapasowe mają wartość tylko wtedy, gdy można z nich skutecznie odzyskać dane w sytuacji kryzysowej.
\end{sphinxadmonition}

\sphinxAtStartPar
\sphinxstylestrong{Kompleksowa strategia backup} powinna obejmować:
\begin{enumerate}
\sphinxsetlistlabels{\arabic}{enumi}{enumii}{}{.}%
\item {} 
\sphinxAtStartPar
Tworzenie kopii zapasowych

\item {} 
\sphinxAtStartPar
Regularne testy restore

\item {} 
\sphinxAtStartPar
Dokumentację procedur

\item {} 
\sphinxAtStartPar
Szkolenie personelu odpowiedzialnego za zarządzanie bazami danych

\end{enumerate}


\chapter{Indeksy i tabele}
\label{\detokenize{index:indeksy-i-tabele}}\begin{itemize}
\item {} 
\sphinxAtStartPar
\DUrole{xref,std,std-ref}{genindex}

\item {} 
\sphinxAtStartPar
\DUrole{xref,std,std-ref}{modindex}

\item {} 
\sphinxAtStartPar
\DUrole{xref,std,std-ref}{search}

\end{itemize}



\renewcommand{\indexname}{Indeks}
\printindex
\end{document}